% Contexto y Requerimientos AgroSense Tech - IEEEtran
\documentclass[conference]{IEEEtran}
\usepackage[utf8]{inputenc}
\usepackage[T1]{fontenc}
\usepackage[spanish]{babel}
\usepackage{hyperref}
\usepackage{booktabs}
\usepackage{tabularx}
\usepackage{adjustbox}
\usepackage{xcolor}
\usepackage{microtype}

\title{Contexto, Buyer Persona y Requerimientos Técnicos del Proyecto AgroSense Tech}
\author{
\IEEEauthorblockN{Alejandro Ramos Echeverry, Luis Alejandro Ojeda Rojas}
\IEEEauthorblockA{\textit{Facultad de Ingenierías} \\
\textit{Universidad de San Buenaventura, Medellín, Colombia} \\
\texttt{alejandro.ramos231@tau.usbmed.edu.co}, \texttt{luis.ojeda222@tau.usbmed.edu.co}}
}

\begin{document}
\maketitle
\pagestyle{empty}

% =============================
% 1. Descripción del Producto
% =============================
\section{Descripción del Producto}
\textbf{AgroSense Tech} es una plataforma IoT para la gestión inteligente de viveros agrícolas.  
Permite el monitoreo, control y análisis en tiempo real de variables como temperatura, humedad, luz y pH mediante sensores conectados a una red IoT.  
Los datos son procesados a través de una \textit{data pipeline} que almacena, transforma y analiza la información para ser visualizada en un dashboard de inteligencia de negocios (BI), con alertas automáticas y análisis predictivo basados en métricas clave del cultivo.

% =============================
% 2. Buyer Persona
% =============================
\section{Buyer Persona}
El cliente ideal es un administrador de vivero o productor agrícola de mediana o gran escala, con conocimientos básicos en tecnología, que busca optimizar el uso de recursos, reducir pérdidas y aumentar la productividad mediante herramientas digitales accesibles y analíticas.

% =============================
% 3. Objetivo del Sistema
% =============================
\section{Objetivo del Sistema}
Desarrollar una solución integral que combine sensorización IoT, procesamiento de datos en tiempo real, análisis visual interactivo y automatización de pruebas de software bajo estándares internacionales de calidad (ISO/IEC 25010, ISO/IEC 29119 e ISTQB).

% =============================
% 4. Requerimientos Técnicos
% =============================
\section{Requerimientos Técnicos}
\subsection{Herramientas}
\begin{itemize}
  \item \textbf{Backend y API:} FastAPI (servicios REST: /sensor-data, /analytics, /dashboard/view), Python 3.11/3.12, Pydantic, Jinja2, estructura modular con routers y modelos.
  \item \textbf{Datos y persistencia:} SQLAlchemy (ORM y sesión de base de datos), PostgreSQL (motor principal), SQLite (modo prueba), helpers de conexión.
  \item \textbf{Pruebas y calidad:} pytest + pytest-cov (unitarias, integración, sistema/E2E), fixtures, niveles ISTQB (unit, integration, system, acceptance), trazabilidad ISO/IEC 29119-3.
  \item \textbf{CI/CD y automatización:} GitHub Actions (matriz Python 3.11/3.12), validaciones de cobertura y artefactos HTML.
  \item \textbf{Documentación técnica:} LaTeX (IEEEtran), paquetes: hyperref, booktabs, tabularx, adjustbox, microtype, xcolor.
  \item \textbf{Diagramas y arquitectura:} Mermaid (.mmd) para representar componentes, flujos y servicios (diagrama\_arquitectura.mmd, web\_application\_architecture.mmd).
  \item \textbf{Entorno de desarrollo:} VS Code, PowerShell, Overleaf, requirements.txt.
\end{itemize}

\subsection{Conceptos}
\begin{itemize}
  \item Diseño de bases de datos relacionales.
  \item Procesamiento de datos IoT en tiempo real.
  \item Pruebas unitarias, funcionales e integración.
  \item Ingeniería de software bajo ISO/IEC 25010 e ISO/IEC 29119.
  \item Automatización de pruebas con Pytest y CI/CD.
  \item Renderización dinámica con plantillas Jinja2.
  \item Criterios de entrada, salida y suspensión según ISO/IEC 29119.
  \item Evidencia objetiva de calidad (cobertura, CI/CD, resultados de pruebas).
\end{itemize}

% =============================
% 5. Estándares y Referencias
% =============================
\section{Estándares y Referencias}
El proyecto se rige por los siguientes estándares internacionales:
\begin{itemize}
  \item \textbf{ISO/IEC 25010:2011} — Modelo de calidad de producto y software.
  \item \textbf{ISO/IEC 29119:2013} — Ingeniería del software — Pruebas de software.
  \item \textbf{ISTQB® Foundation Level v4.0} — Técnicas y niveles de prueba.
  \item \textbf{IEEEtran} — Formato para reportes técnicos y académicos.
\end{itemize}

% =============================
% 6. Conclusión
% =============================
\section{Conclusión}
Este documento complementa el plan y reporte de pruebas de \textbf{AgroSense Tech}, estableciendo el contexto del producto, su usuario objetivo y la infraestructura tecnológica empleada.  
Proporciona la base conceptual y técnica para la verificación de calidad del software, la trazabilidad de los requisitos y la aplicación de buenas prácticas en ingeniería de datos e IoT bajo normas internacionales.

\end{document}
