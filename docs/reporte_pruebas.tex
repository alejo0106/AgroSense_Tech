% Compilable on Overleaf (pdfLaTeX) — IEEEtran format
\documentclass[conference]{IEEEtran}

% Encoding and language
\usepackage[utf8]{inputenc}
\usepackage[T1]{fontenc}
\usepackage[spanish,english]{babel}
\usepackage{csquotes}
\usepackage{hyperref}
\hypersetup{colorlinks=true, linkcolor=black, citecolor=black, urlcolor=blue}

% Tables and layout
\usepackage{booktabs}
\usepackage{tabularx}
\usepackage{longtable}
\usepackage{array}
\usepackage{multirow}
\usepackage{url}

% Prevent table overflow
\usepackage{adjustbox}

% For symbols and better spacing
\usepackage{microtype}

% Precise float placement and float barriers
\usepackage{float}
\usepackage{placeins}

\title{Reporte de pruebas: AgroSense Tech}

\author{\IEEEauthorblockN{1\textsuperscript{st} Luis Alejandro Ojeda}
\IEEEauthorblockA{\textit{Facultad de Ingenier\'ias} \\
  	extit{Universidad de San Buenaventura}\\
Medell\'in, Colombia \\
luis.ojeda222@tau.usbmed.edu.co}
\and
\IEEEauthorblockN{2\textsuperscript{nd} Alejandro Ramos Echeverry}
\IEEEauthorblockA{\textit{Facultad de Ingenier\'ias} \\
  	extit{Universidad de San Buenaventura}\\
Medell\'in, Colombia \\
alejandro.ramos231@tau.usbmed.edu.co}
}

\begin{document}
\selectlanguage{spanish}
\maketitle

% Abstract in English
\begin{otherlanguage*}{english}
\begin{abstract}
This document presents the professional software testing report for the AgroSense\_Tech project, an IoT-enabled FastAPI application for agricultural sensing and analytics. The test process follows the ISTQB body of knowledge and conforms to ISO/IEC 29119, while the interpretation of results is mapped to ISO/IEC 25010 quality attributes. A multi-level test suite (unit, integration, and system) was implemented covering database access (SQLAlchemy/PostgreSQL), API endpoints, analytics consistency, and dashboard rendering. The continuous integration pipeline executes the test matrix and aggregates coverage. The final coverage achieved is 91\% overall, with no critical failures detected across 26 automated tests. The report summarizes the plan, strategy, evidence, and quality assessment to support release decisions and future extensions (e.g., IoT simulator and performance testing).
\end{abstract}
\end{otherlanguage*}

% Resumen en español
\section*{Resumen}
Este documento presenta el reporte profesional de pruebas de software del proyecto AgroSense\_Tech, una aplicaci\'on FastAPI con capacidades IoT para sensado y anal\'itica en agricultura. El proceso de pruebas se bas\'o en ISTQB y se aline\'o al est\'andar ISO/IEC 29119, mientras que la interpretaci\'on de resultados se mape\'o a los atributos de calidad del modelo ISO/IEC 25010. Se implement\'o una bater\'ia multi-nivel (unitarias, integraci\'on y sistema) que cubre acceso a datos (SQLAlchemy/PostgreSQL), endpoints de API, consistencia de anal\'iticas y render del dashboard. La integraci\'on continua ejecuta la matriz de pruebas y agrega la cobertura. La cobertura final alcanzada es 91\% a nivel global, sin fallos cr\'iticos en 26 pruebas automatizadas. El reporte resume plan, estrategia, evidencias y evaluaci\'on de calidad para soportar decisiones de liberaci\'on y extensiones futuras (p.ej., simulador IoT y pruebas de desempe\~no).

\section{Introducci\'on}
AgroSense\_Tech es un sistema de an\'alisis para datos de sensores agr\'icolas construido con FastAPI, SQLAlchemy, PostgreSQL (con fallback SQLite en pruebas) y un dashboard HTML. El objetivo del proceso de prueba fue validar adecuaci\'on funcional, fiabilidad, mantenibilidad y eficiencia, centrándose en flujos de ingesta, anal\'itica y visualizaci\'on.

El reporte sigue una estructura estándar de ingeniería de pruebas: resumen ejecutivo, marco normativo, plan, estrategia, resultados y conclusiones. A continuación se presentan las definiciones normativas antes de su aplicación explícita.

\section{Marco Normativo y Conceptual}
\subsection{ISTQB}
ISTQB (International Software Testing Qualifications Board) mantiene un cuerpo de conocimiento que estandariza principios (exhaustividad imposible, prueba depende del contexto, agrupaci\'on de defectos), niveles (unitaria, integraci\'on, sistema, aceptaci\'on), roles y técnicas (caja negra, caja blanca, basadas en experiencia, partición de equivalencia, análisis de valores límite). Proporciona terminología uniforme que reduce ambigüedad en diseño y reporte.
\subsection{ISO/IEC 25010}
Define el modelo de calidad de producto software en ocho características: adecuaci\'on funcional, eficiencia de rendimiento, compatibilidad, usabilidad, fiabilidad, seguridad, mantenibilidad y portabilidad. Cada característica puede desglosarse en subatributos (p. ej. mantenibilidad: modularidad, reusabilidad, analizabilidad, modificabilidad, testabilidad) y se evalúa mediante evidencia objetiva (métricas, resultados de pruebas, cobertura).
\subsection{ISO/IEC 29119}
Estándar internacional para procesos de prueba que estructura actividades en: planificación, monitoreo/control, análisis y diseño, implementación, ejecución y cierre. Define artefactos (plan, especificaciones de casos, procedimientos, informes), vocabulario y criterios de entrada/salida para asegurar repetibilidad y trazabilidad.
\subsection{Aplicaci\'on en AgroSense Tech}
La selección de pruebas unitarias sobre helpers y modelos respalda \textit{mantenibilidad} y \textit{fiabilidad} (ISO/IEC 25010) al detectar regresiones tempranas. Pruebas de integraci\'on sobre routers y base de datos validan \textit{adecuaci\'on funcional} y consistencia de datos. El flujo end-to-end garantiza que las métricas agregadas reflejan correctamente la ingesta (fiabilidad y adecuación). El uso de fixtures y separación en m\'odulos promueve \textit{modularidad} y \textit{testabilidad}. La cobertura del 91\% cuantifica soporte a la \textit{mantenibilidad}. El proceso seguido mapea a ISO/IEC 29119: (1) Plan (sección correspondiente), (2) Diseño (tabla de casos), (3) Implementación (scripts pytest y fixtures), (4) Ejecución (CI multi-versión), (5) Evaluación y cierre (sección de resultados y conclusiones). La terminología de niveles/técnicas utilizada proviene de ISTQB, facilitando comunicación clara y extensible.

\section{Plan de Pruebas (ISO/IEC 29119)}
\subsection{Tipos de pruebas aplicadas}
Se ejecutaron los siguientes niveles/tipos:
\begin{itemize}
  \item Pruebas Unitarias: validaci\'on de funciones internas (p. ej., procesamiento de datos, utilidades de base de datos, controladores de ingesta).
  \item Pruebas de Integraci\'on: interacci\'on entre m\'odulos (rutas/routers y capa de datos), coherencia de endpoints y respuesta del dashboard JSON/HTML.
  \item Pruebas de Sistema: verificaci\'on extremo a extremo del flujo de ingesta y el reflejo de las m\'etricas en anal\'iticas.
  \item Pruebas de Aceptaci\'on: planificadas como validaci\'on por partes interesadas con criterios funcionales y de usabilidad; no ejecutadas en este corte.
\end{itemize}

\subsection{Etapas y criterios de entrada/salida}
Criterios de entrada:\\
- Entorno de CI operativo con dependencias del sistema (libpq, gcc) y matrices de Python (3.11/3.12).\\
- Variables de entorno y configuraci\'on de base de datos disponibles (.env/DATABASE\_URL, fallback a SQLite).\\
- Suite de pruebas organizada y fixtures de cliente y sesi\'on de DB funcionales.

Criterios de salida:\\
- 100\% de los casos definidos ejecutados y pasados en el pipeline.\\
- Cobertura total \textgreater= 80\% (lograda: 91\%).\\
- Sin defectos cr\'iticos abiertos en rutas de valor (ingesta, anal\'itica, dashboard).

\subsection{Entorno de pruebas}
Las pruebas se ejecutaron sobre Ubuntu 22.04, Python 3.12, FastAPI 0.115, SQLAlchemy 2.0 y PostgreSQL 16. El pipeline CI utiliz\'o GitHub Actions con matrices Python 3.11 y 3.12, bajo entorno controlado con dependencias validadas.

\subsection{Tabla resumen de casos de prueba}
La Tabla~\ref{tab:plan} sintetiza el plan (fuente: tests\_report\_plan\_ES.md).

\begin{table*}[t]
\centering
\caption{Resumen de Casos de Prueba} \label{tab:plan}
\small
\begin{adjustbox}{max width=\textwidth}
\begin{tabularx}{\textwidth}{>{\raggedright\arraybackslash}p{2.2cm} X >{\raggedright\arraybackslash}p{3.2cm} >{\raggedright\arraybackslash}p{2.2cm}}
\toprule
\textbf{ID} & \textbf{Nombre de la Prueba} & \textbf{Tipo} & \textbf{Estado} \\
\midrule
CP-SENS-01 & test\_sensor\_data\_post\_valid & Prueba Unitaria & Completada \\
CP-SENS-02 & test\_sensor\_data\_post\_invalid & Prueba Unitaria & Completada \\
CP-PROC-01 & test\_process\_data\_correctness & Prueba Unitaria & Completada \\
CP-PROC-02 & test\_process\_data\_empty & Prueba Unitaria & Completada \\
CP-AN-01 & test\_analytics\_endpoint\_consistency & Prueba de Integraci\'on & Completada \\
CP-DB-01 & test\_dashboard\_load & Prueba de Integraci\'on & Completada \\
CP-DB-02 & test\_dashboard\_empty\_state & Prueba de Integraci\'on & Completada \\
CP-E2E-01 & test\_end\_to\_end\_ingest\_and\_metrics & Prueba de Sistema & Completada \\
CP-LEG-01 & test\_placeholder\_cleanup (analytics legacy) & Legado & Conservado \\
CP-LEG-02 & test\_placeholder\_cleanup\_dashboard & Legado & Conservado \\
CP-LEG-03 & test\_placeholder\_cleanup\_sensor & Legado & Conservado \\
CP-LEG-04 & test\_placeholder\_cleanup\_sensors\_dup & Legado & Conservado \\
CP-FIX-01 & client fixture available & Fixture de Pruebas & Completada \\
CP-FIX-02 & db\_session fixture cleanup & Fixture de Pruebas & Completada \\
CP-COV-01 & Coverage $\geq$ 80\% core modules & Calidad & Completada (91\% total) \\
CP-COV-02 & Add coverage for database helpers & Calidad & Completada (database.py 88\%) \\
CP-COV-03 & Add coverage for sensor\_simulator script & Calidad & Pendiente \\
\bottomrule
\end{tabularx}
\end{adjustbox}
\end{table*}

\FloatBarrier

\section{Estrategia de Pruebas (ISTQB)}
La estrategia combin\'o t\'ecnicas de caja negra, caja blanca y basadas en requisitos:
\begin{itemize}
  \item Caja negra: validaci\'on de endpoints REST (ingesta, anal\'iticas, dashboard JSON/HTML) con partici\'on de equivalencia y valores l\'imite.
  \item Caja blanca: cobertura de funciones internas (procesamiento de datos, manejo de errores y transacciones en capa de datos, helper de conexi\'on psycopg2) y rutas de c\'odigo cr\'iticas.
  \item Basadas en requisitos: trazabilidad de casos a funcionalidades clave (ingesta de sensores, c\'alculo de m\'etricas, render de tablero), priorizando adecuaci\'on funcional y fiabilidad.
\end{itemize}
La selecci\'on se ajusta a la arquitectura: FastAPI para la capa de presentaci\'on/servicios, SQLAlchemy para ORM, PostgreSQL como base transaccional (con fallback SQLite) y Plotly/HTML para visualizaci\'on. Las fixtures de \textit{pytest} (cliente y sesi\'on de BD) permiten pruebas aisladas y repetibles.

\section{Pipeline CI/CD en GitHub Actions}
El flujo de Integraci\'on Continua (CI) se implementa en GitHub Actions usando una matriz de versiones de Python (3.11 y 3.12). Las fases principales incluyen: (1) instalaci\'on de dependencias (FastAPI, SQLAlchemy, drivers de BD, pytest y coverage), (2) ejecuci\'on de pruebas con \texttt{pytest --cov=.}, (3) c\'alculo y agregaci\'on de cobertura (HTML/XML) y (4) publicaci\'on opcional de artefactos (reporte HTML de cobertura y logs). Condiciones de paso: todas las pruebas deben finalizar sin fallos cr\'iticos y una cobertura global \(\geq 80\%\). Beneficios: detecci\'on temprana de regresiones, reproducibilidad en entornos limpios, evidencia de proceso seg\'un ISO/IEC 29119 y soporte a fiabilidad y mantenibilidad (ISO/IEC 25010).

\section{Aplicaci\'on del modelo ISO/IEC 25010 en el proyecto}
\begin{itemize}
  \item \textbf{Adecuaci\'on funcional}: Endpoints de ingesta (POST /sensor-data), anal\'itica (GET /analytics) y dashboard (GET /dashboard/view) validados por CP-SENS, CP-AN, CP-DB y CP-E2E; confirman cobertura de funciones clave y resultados esperados.
  \item \textbf{Fiabilidad}: 26 pruebas sin fallos cr\'iticos; casos de vac\'io (CP-PROC-02, CP-DB-02) y consistencia de c\'alculos verifican comportamiento estable ante bordes.
  \item \textbf{Mantenibilidad}: Arquitectura modular (routers, modelos, capa BD), alta cobertura (91\%), fixtures reutilizables y CI frecuente mejoran analizabilidad/testabilidad y reducen costo de cambio.
  \item \textbf{Eficiencia de rendimiento}: A\'un sin pruebas de carga, las rutas realizan operaciones simples (agregaciones y render ligero); base lista para incluir perfiles y monitoreo en simulador IoT.
  \item \textbf{Compatibilidad}: Interfaces REST est\'andar y JSON; integraciones internas (routers+BD) ejercitadas en pruebas de integraci\'on, facilitando clientes externos.
  \item \textbf{Usabilidad}: Dashboard HTML validado (CP-DB-01/02) para estados normales y vac\'ios; presenta m\'etricas agregadas de forma legible para usuario t\'ecnico.
  \item \textbf{Seguridad}: Enfoque inicial en validaciones de entrada mediante modelos; preparado para incorporar autenticaci\'on/autorizaci\'on y pruebas negativas en ciclo siguiente.
  \item \textbf{Portabilidad}: FastAPI + SQLAlchemy independiza del motor (PostgreSQL/SQLite); configuraci\'on por entorno permite despliegue multi-plataforma.
\end{itemize}

\subsection*{M\'etricas de Cobertura por M\'odulo}
\begin{table}[H]
\centering
\caption{M\'etricas de Cobertura por M\'odulo}
\small
\begin{tabular}{lcc}
	\toprule
	\textbf{M\'odulo} & \textbf{Cobertura} & \textbf{Tipo de Pruebas} \\
\midrule
models.py & 100\% & Unitarias \\
database.py & 88\% & Unitarias \\
routers.analytics.py & 100\% & Integraci\'on \\
routers.dashboard\_html.py & 89\% & Integraci\'on \\
main.py & 91\% & Sistema \\
\bottomrule
\end{tabular}
\end{table}

\FloatBarrier

\section{Resultados y Cobertura}
Cobertura global consolidada: \textbf{91\%} sobre 26 pruebas sin fallos cr\'iticos. La ausencia de errores en ingesta, procesamiento y visualizaci\'on valida la fiabilidad en el alcance actual. La diversidad de niveles (unitaria, integraci\'on, sistema) demuestra profundidad y apoya decisiones para extender a pruebas no funcionales (rendimiento, seguridad) en ciclos siguientes.

\section{Conclusiones}
El proyecto cumple con los lineamientos de calidad de ISO/IEC 25010 y el proceso de pruebas de ISO/IEC 29119, aplicando una estrategia ISTQB que valid\'o las funcionalidades cr\'iticas de ingesta, anal\'itica y visualizaci\'on. Se alcanz\'o una cobertura superior al 80\% (91\% global), con evidencias de fiabilidad y mantenibilidad.

Elementos pendientes para ciclos posteriores: (i) pruebas del simulador IoT, actualmente marcadas como \textit{Pendiente}; (ii) pruebas de desempe\~no/estr\'es y umbrales de cobertura en CI; (iii) integraci\'on con servicios de reporte de cobertura (Codecov/Coveralls) y/o migraciones de esquema (Alembic) para gesti\'on de cambios.

\section{Mejoras propuestas}
Con base en ISO/IEC 29119 e incrementando la madurez del proceso se proponen:
\begin{itemize}
  \item \textbf{Pruebas de estr\'es y carga}: Simular ráfagas de ingest\'on desde un simulador IoT para medir tiempos de respuesta, colas y uso de recursos.
  \item \textbf{Automatizaci\'on avanzada}: Pruebas de contrato y escenarios parametrizados para endpoints y consultas, ampliando cobertura de bordes.
  \item \textbf{Integraci\'on con Codecov}: Publicar tendencia de cobertura, fijar umbrales crecientes (91\% \rightarrow 95\%) y proteger ramas.
  \item \textbf{Monitoreo del simulador IoT}: M\'etricas de salud (latencia, tasa de eventos procesados) y alertas para regresiones.
  \item \textbf{Seguridad}: Incorporar autenticaci\'on por token y pruebas de autorizaci\'on negativas (respuestas 401/403).
  \item \textbf{Pruebas de aceptaci\'on}: Criterios de usuario final para el dashboard (claridad, tiempos de carga) y endpoints clave.
\end{itemize}

\section{Evidencias}
\begin{itemize}
\item Reporte HTML de cobertura (\texttt{htmlcov/index.html}).
\item Capturas de ejecuci\'on de pruebas (\texttt{pytest --cov=.}).
\item Diagrama de arquitectura (\texttt{docs/diagrama\_arquitectura.mmd} exportado a PNG).
\end{itemize}

\section*{Referencias}
\begin{thebibliography}{00}
\bibitem{iso25010} ISO/IEC 25010:2011, \textit{Systems and software engineering — Systems and software Quality Requirements and Evaluation (SQuaRE) — System and software quality models}.
\bibitem{iso29119} ISO/IEC/IEEE 29119:2013, \textit{Software and systems engineering — Software testing}.
\bibitem{istqb} ISTQB\textregistered{} Foundation Level Syllabus v4.0.
\bibitem{fastapi} FastAPI Documentation: \url{https://fastapi.tiangolo.com/}
\bibitem{sqlalchemy} SQLAlchemy Documentation: \url{https://www.sqlalchemy.org/}
\bibitem{pytest} Pytest Documentation: \url{https://docs.pytest.org/}
\end{thebibliography}

\end{document}
